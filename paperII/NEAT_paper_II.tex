% mn2esample.tex
%
% v2.1 released 22nd May 2002 (G. Hutton)
%
% The mnsample.tex file has been amended to highlight
% the proper use of LaTeX2e code with the class file
% and using natbib cross-referencing. These changes
% do not reflect the original paper by A. V. Raveendran.
%
% Previous versions of this sample document were
% compatible with the LaTeX 2.09 style file mn.sty
% v1.2 released 5th September 1994 (M. Reed)
% v1.1 released 18th July 1994
% v1.0 released 28th January 1994

\documentclass[useAMS,usenatbib]{../paper/mn2e}
\usepackage{graphicx}
\usepackage{../paper/aas_macros}

% If your system does not have the AMS fonts version 2.0 installed, then
% remove the useAMS option.
%
% useAMS allows you to obtain upright Greek characters.
% e.g. \umu, \upi etc.  See the section on "Upright Greek characters" in
% this guide for further information.
%
% If you are using AMS 2.0 fonts, bold math letters/symbols are available
% at a larger range of sizes for NFSS release 1 and 2 (using \boldmath or
% preferably \bmath).
%
% The usenatbib command allows the use of Patrick Daly's natbib.sty for
% cross-referencing.
%
% If you wish to typeset the paper in Times font (if you do not have the
% PostScript Type 1 Computer Modern fonts you will need to do this to get
% smoother fonts in a PDF file) then uncomment the next line
% \usepackage{Times}

%%%%% AUTHORS - PLACE YOUR OWN MACROS HERE %%%%%


%%%%%%%%%%%%%%%%%%%%%%%%%%%%%%%%%%%%%%%%%%%%%%%%

\title[Systematic uncertainties in nebular abundances]{Systematic uncertainties in nebular abundance determinations} %have to be the same?
\author[R. Wesson et al.]{R. Wesson$^{1,2}$, D.J. Stock$^{1,3}$ \& P. Scicluna$^{1,4}$.  Bruce?\\
$^1$Department of Physics and Astronomy, University College London, Gower Street, London WC1E 6BT, UK\\
$^2$European Southern Observatory, Alonso de C\'ordova 3107, Casilla 19001, Santiago, Chile\\
$^3$Department of Physics and Astronomy, University of Western Ontario, London, Ontario, Canada, N6K 3K7\\
$^4$European Southern Observatory, Karl-Schwarzschildstr. 2, 85748 Garching, Germany\\ 
}


\begin{document}

\date{}

\pagerange{\pageref{firstpage}--\pageref{lastpage}} \pubyear{2002}

\maketitle

\label{firstpage}

\begin{abstract}

Determinations of abundances from emission line nebulae are of crucial importance in understanding many astronomical environments.  Many methodological choices must be made before abundances can be calculated, including the choice of interstellar extinction law, the atomic data to be used, and the ionisation correction scheme.  Naturally one needs to know if the methodology chosen matters, and if so, how much and when?  This question has only been partially answered in the literature and for many methodological choices, personal preference is the deciding factor.

In Wesson et al. 2012, we developed a method of uncertainty propagation which accounted for non-gaussian uncertainties and avoided approximations which may break down in real astronomical data.  This method allows us to accurately quantify statistical uncertainties in abundance determinations.  We now use our code to analyse a large sample of nebulae, using 5 extinction laws, 4 sets of heavy element collisional data, 2 sets of helium atomic data, and 2 ICFs.  By comparing the differences between all possible permutations of methodological choice with the statistical uncertainties, we answer the question.

\end{abstract}

\begin{keywords}
ISM: abundances -- atomic processes -- methods: statistical
\end{keywords}

\section{Introduction}

Outline of empirical method\\
Brief discussion of different atomic data, extinction laws, ICFs\\

\section{Summary of Paper I}

Propagation of statistical uncertainties, reliability of MC method as opposed to analytical techniques\\
Description of updates to code since Paper I.
Mention the correction to the Ar ICF which was wrongly given in KB94?

\section{Sample nebulae}

Reasons for selection (if we have any beyond "they were there").\\
Plots showing histograms of extinction, excitation class, temperatures, densities, O/H\\

\section{Results}

Results for each parameter in turn, reddening, atomic data, ICF.  How much of a difference does each one make?\\
Can we quantify this in a simple way by saying for example, for O/H, that the "effect" is the ratio of the amount a change of methodology changes the abundance by to the statistical uncertainty?  For example, if O/H in the reference case is found to be 1e-4 +- 1e-5, and using a different extinction law makes it 1.4e-4 +- 1e-5, then the "effect" is 4x the statistical uncertainty.
Presenting these results clearly and effectively is crucial.

\section{Discussion}

What have we found?

\section*{Acknowledgments}

This work was co-funded under the Marie Curie Actions of the European Commission (FP7-COFUND). DJS is supported by an NSERC Discovery Accelerator Grant.  PS is supported by the ESO Studentship Programme.

\bibliographystyle{mn2e}
\bibliography{NEAT_paper}

\label{lastpage}

\end{document}
